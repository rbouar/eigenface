\documentclass{beamer}

\usepackage[french]{babel}
\usepackage[T1]{fontenc}
\usepackage[utf8]{inputenc}

\usepackage{amsmath}
\usepackage{amssymb}
\usepackage{amsthm}


% quelques définitions
\theoremstyle{plain}
\newtheorem{thm}{Théorème}
\newtheorem{cor}[thm]{Corollaire}
\newtheorem{lem}[thm]{Lemma}
\newtheorem{prop}{Proposition}
\newtheorem{dem}{Démonstration}

\theoremstyle{definition}
\newtheorem{defi}{Définition}
\newtheorem{qst}{Question}


\usetheme{Berlin}

\title{Reconnaissance faciale par Eigenfaces}
\author{Bouarah Romain \and Langdorph Matthieu \and Ketels Lucas \and Nathan Souffan }
\date{\today}

\begin{document}


\begin{frame}[plain]
  \titlepage
\end{frame}


\begin{frame}[shrink]
  \tableofcontents
\end{frame}


\section{Introduction}
\subsection{Motivation}
\subsection{Histoire}

\section{Calcul des eigenfaces}
\subsection{Travail dans $\mathbb{R}^{N \times N}$}
\begin{frame}
  \frametitle{Représentation matricielle des images}
  \begin{defi}
    Une image de taille $N \times N$ est représentée par une matrice $N \times N$.\\
    Chaque coefficient représente un niveau de gris d'un pixel.
  \end{defi}
\end{frame}

\begin{frame}
  \frametitle{Transformation en un vecteur de $\mathbb{R}^{N \times N}$}
  On juxtapose simplement les colonnes de la matrice l'une en dessous de l'autre.
  \[
    \begin{pmatrix}
      p_{1,1} & p_{1,2} & \cdots & p_{1,N} \\
      p_{2,1} & p_{2,2} & \cdots & p_{2,N} \\
      \vdots  & \vdots  & \ddots & \vdots  \\
      p_{N,1} & p_{N,2} & \cdots & p_{N,N}
    \end{pmatrix}
    \rightarrow
    \begin{pmatrix}
      p_{1,1} \\
      p_{2,1} \\
      \vdots \\
      p_{N,1} \\
      \vdots \\
      p_{1,N} \\
      \vdots \\
      p_{N,N}
    \end{pmatrix}
  \]
  
\end{frame}



\subsection{Matrice de covariance}

\begin{frame}  
  \frametitle{Observation sur les images des visages}
  \begin{qst}
    Que dire de la position de nos images de visage dans l'espace $\mathbb{R}^{N \times N}$?
  \end{qst}
  \pause
  \begin{exampleblock}{Réponse}
    Nos images de visages ne sont pas si éloignées les unes des autres. 
  \end{exampleblock}
\end{frame}


\begin{frame}
\begin{defi}[Matrice de Covariance]
  La matrice de covariance d'un vecteur de $p$ variables aléatoires $\overrightarrow{X} =
  \begin{pmatrix}
    X_1 \\
    \vdots \\
    X_p
  \end{pmatrix}$ dont chacune possède une variance, est la matrice carrée dont le terme générique est donné par $a_{i,j} = Cov(X_i,X_j)$.
\end{defi}

\begin{defi}[Matrice de Covariance]
  La matrice de covariance est définie par $Var(\overrightarrow{X}) = \mathrm{E}[ (\overrightarrow{X}-\mathrm{E}(\overrightarrow{X})) (\overrightarrow{X}-\mathrm{E}(\overrightarrow{X}))^T]$
\end{defi}
\end{frame}


\subsection{Analyse en composantes principales}
\subsection{Décomposition en valeurs singulières}

\section{Classification des visages}
\subsection{Projection dans l'espace des visages}
\subsection{Analyse de la projection}

\section{Application}
\subsection{Techniques utilisées aujourd'hui}
\subsection{Par les téléphones}

\section{Conclusion}
\subsection{Risque de la reconnaissance faciale}
\subsection{Références}

\end{document}