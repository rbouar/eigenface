\documentclass[12pt,french]{article}

\usepackage{amssymb}
\usepackage{amsmath}
\usepackage{babel}

\title{Eigenfaces}
\author{
  Bouarah Romain \and
  Langdorph Matthieu \and
  Ketels Lucas \and
  Souffan Nathan
}


\begin{document}
\maketitle
\newpage

\part{Partie Mathématiques}


\section{Calcul des Eigenfaces}

\subsection{Travail dans $\mathbb{R}^{N \times N}$}
Considérons une image de visage comme une matrice $N \times N$ dont le coefficient $(i,j)$ est égal au niveau de gris du pixel $(i,j)$ (l'origine se situant dans le coin haut gauche).
On transforme ensuite cette matrice comme un vecteur de $\mathbb{R}^{N \times N}$ en juxtaposant les colonnes l'une en dessous de l'autre, par exemple.
\[
  \begin{pmatrix}
    p_{1,1} & p_{1,2} & \cdots & p_{1,N} \\
    p_{2,1} & p_{2,2} & \cdots & p_{2,N} \\
    \vdots  & \vdots  & \ddots & \vdots  \\
    p_{N,1} & p_{N,2} & \cdots & p_{N,N}
  \end{pmatrix}
  \rightarrow
  \begin{pmatrix}
    p_{1,1} \\
    p_{2,1} \\
    \vdots \\
    p_{N,1} \\
    \vdots \\
    p_{1,N} \\
    \vdots \\
    p_{N,N}
  \end{pmatrix}
\]

\subsection{Calcul des valeurs propres et des vecteurs propres de la matrice de covariance}
Les images des visages sont globalement similaires, donc ces images ne seront pas distribuées aléatoirement dans notre espace $\mathbb{R}^{N \times N}$.
On peut donc décrire notre espace des visages de manière plus fine (\textit{i.e.} avec moins de dimensions).

\subsubsection{Matrice de covariance}
Avant de commencer la réduction de l'espace de travail, nous allons d'abord calculer la matrice de covariance.\\
Supposons que nous avons $M$ images de visage qu'on note $\Gamma_1,~\Gamma_2,~\dots,~\Gamma_M$. On a $\Psi = \frac{1}{M}\displaystyle\sum_{i=1}^{M} \Gamma_i$ correspondant à la moyenne des visages.
Chaque visage différe donc de la moyenne par le vecteur $\Phi_i = \Gamma_i - \Psi$.
\\ \emph{Définir la matrice de covariance}\\
La matrice de covariance est $C = \frac{1}{M}\displaystyle\sum_{i=1}^{M} \Phi_i \Phi_i^T=AA^T$ où la matrice $A = [\Phi_1 \Phi_2 \dots \Phi_M]$

\subsubsection{Méthode 1: Analyse en composantes principales}
\subsubsection{Méthode 2: Décomposition en valeurs singulières}
\section{Utilisation des eigenfaces pour classer une image de visage}
% Les visages calculés avec les vecteurs propres forment un ensemble à l'aide duquel nous pouvons décrire un visage. \\
% 40 eigenfaces suffisent pour avoir une bonne description de l'ensemble des visages
\emph{Définir $M'$ comme étant le nombre de visage du training set}\\
\emph{Définir la base de l'espaces des eigenfaces}

\subsection{Projection dans l'espace des visages}
Soit $\Gamma$ une nouvelle image de visage, on la projette dans l'espace des visages par:
$$\omega_k = u_k^T(\Gamma - \Psi)$$
pour $k = 1,~\dots,~M'$, on obtient ainsi un vecteur $\Omega$ tel que:
\[\Omega =
\begin{pmatrix}
  \omega_1 \\
  \vdots \\
  \omega_{M'}
\end{pmatrix}
\]
$\Omega$ décrit la contribution de chacun des eigenfaces pour l'image en question.

\subsection{Analyse de la projection}
On peut maintenant utiliser ce vecteur pour reconnaître si ce vecteur correspond à un visage déjà connue étant dans la base ou si c'est un visage inconnu.
Pour ce faire, on cherche la classe $k$ qui minimise la distance euclidienne $\epsilon_k = \|(\Omega - \Omega_k)\|^2$
 où $\Omega_k$ est le vecteur décrivant la $k^{ieme}$ classe de visage. Les classes de visages $\Omega_i$ sont calculés en faisant la moyenne de plusieurs ou une image du visage de chaque individu. \\
 On considère ensuite qu'un visage appartient à une certaine classe de visage $k$ si le minimum $\epsilon_k$ est en dessous un certain seuil $\Theta$.
 Si $\forall k,~\epsilon_k > \Theta$, alors le visage est inconnu et on a alors une nouvelle classe de visage. \\
 Il y a finalement 4 possibilités pour une image:
\begin{enumerate}
  \item L'image est proche de l'espace des visages et proche d'une classe de visage en particulier, c'est alors un visage connu
  \item L'image est proche de l'espace des visages mais n'est proche d'aucune classe de visage, c'est alors un visage inconnu
  \item L'image est distante de l'espace des visages mais proche d'une classe de visage, \emph{EXPLICATION!!!} C'est la plupart du temps un faux positif.
  \item L'image est distante de l'espace des images et également distante de toutes les classes de visages, on peut alors en conlcure que ce n'est pas un visage
\end{enumerate}











\end{document}
