\title{Eigenfaces}
\author{
Bouarah Romain \and
Langdorph Matthieu \and
Ketels Lucas \and
Souffan Nathan
}
\documentclass[12pt]{article}
\usepackage{amssymb}
\begin{document}
\maketitle
\newpage
\section{Partie Mathématiques}
\subsection{Calcul des Eigenfaces}
\subsubsection{Travail dans $\mathbb{R}^{N \times N}$}
Considérons une image de visage comme une matrice $N \times N$ dont le coefficient $(i,j)$ est égal au niveau de gris du pixel $(i,j)$ (l'origine se situant dans le coin haut gauche).  
On transforme ensuite cette matrice comme un vecteur de $\mathbb{R}^{N \times N}$ en juxtaposant les colonnes l'une en dessous de l'autre, par exemple.
\subsubsection{Calcul des valeurs propres et des vecteurs propres de la matrice de covariance}
Les images des visages sont globalement similaires, donc ces images ne seront pas distribuées aléatoirement dans notre espace $\mathbb{R}^{N \times N}$.
On peut donc décrire notre espace des visages de manière plus fine (\textit{i.e.} avec moins de dimensions).
\paragraph{Matrice de covariance}
\paragraph{Méthode 1: Analyse en composantes principales}
\paragraph{Méthode 2: Décomposition en valeurs singulières}
\subsection{Utilisation des eigenfaces pour classer une image de visage}
\subsubsection{Projection dans l'espace des visages}
\subsubsection{Analyse de la projection}









\end{document}
